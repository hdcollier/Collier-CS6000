
\documentclass[11pt]{article}
\usepackage{palatino}
\usepackage{url}
\usepackage{graphicx}
\usepackage{amsmath}%
\usepackage{amsfonts}%
\usepackage{amssymb}%
\usepackage{setspace}
\usepackage{cite}
\usepackage{hyperref}
\usepackage {rotating}
\begin{document}
\begin {center}
\section*{CS6000-Journal 2}
By Henry Collier
\\
9/8/2018
\end {center}

\section*{Process}
My process for reading research papers is similar to what is outlined in the power point, but perhaps a little more abbreviated as I like to use my time as wisely as possible. First I read the title, which often times gives me a clue as to which papers are definitely not ones that I need to pursue. The titles that are vague or ambiguous enough to either not identify themselves as worth to read or not read get a deeper dive to determine their validity to my research interests. After eliminating the papers that are definitely not in my wheelhouse, I then start to read the abstracts to see if the paper is worth of further effort. If the abstract doesn't get my attention within the first handful of lines, the likely hood that it will be of significance to me goes down. However, I read the entire abstract and then determine if further time and effort is worth the expenditure. Having Attention Deficit Disorder makes reading a lot of these papers tiresome, so I do try to limit the papers to the ones that are truly of interest to me. This way I can reduce the chances of becoming distracted and losing track where I am. Seeing that my process is similar to the process outlined in the PowerPoints shows me that I am on the right path, but that I need to work on my process to overcome the issues with my ADD and to become a better researcher.

\section*{Raw Notes }
Raw notes from critical read of Attribute Inference AtTtacks in Online Social Networks by Gong and Liu. 
\\
\begin{turn}{270}
  
\includegraphics[width=100mm]{Paper 2 notes 1.jpg}

\end{turn}
\begin{turn}{270}
  
\includegraphics[width=100mm]{Paper 2 notes 2.jpg}

\end{turn}
\begin{turn}{270}

  
\includegraphics[width=100mm]{Paper 2 notes 3.jpg}

\end{turn}
\begin{turn}{270}

  
\includegraphics[width=100mm]{Paper 2 notes 4.jpg}

\end{turn}
\\
Additional notes: 
Would have liked better examples to show validity of proofs. 
Felt authors could have better shown practicality of their attack. 
This type of threat is serious and is growing; however, I don't see social engineers doing it this way at this point. I belive they would take a less technial approach to conducting this type of attack. 
\\
\\
Raw notes from critical read of Technological and Human Factors of Malware Attacks: A Computer Secrutiy Clinical Trial Approach by Levesque, Chiasson, Somayaji and Fernandez. \\
\begin{turn}{270}
  
\includegraphics[width=100mm]{Paper notes 1.jpg}

\end{turn}
\begin{turn}{270}
  
\includegraphics[width=100mm]{Paper notes 2.jpg}

\end{turn}
\begin{turn}{270}

  
\includegraphics[width=100mm]{Paper notes 3.jpg}

\end{turn}
\begin{turn}{270}

  
\includegraphics[width=100mm]{Paper notes 4.jpg}

\end{turn}
\begin{turn}{270}

  
\includegraphics[width=100mm]{Paper notes 5.jpg}

\end{turn}
\begin{turn}{270}

  
\includegraphics[width=100mm]{Paper notes 6.jpg}

\end{turn}
\begin{turn}{270}

  
\includegraphics[width=100mm]{Paper notes 7.jpg}

\end{turn}


\nocite {*}
\bibliographystyle{ieeetran}
\bibliography{Journal2_References}
\end{document}